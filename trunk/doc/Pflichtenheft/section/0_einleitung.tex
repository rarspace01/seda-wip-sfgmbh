\section{Einleitung}
\label{sec:Einleitung}

\subsection*{(Anna Kupfer)}

Die Vorliegende Arbeit enthält die Gesamtheit an notwendigen Spezifikationen die uns von der Otto-Friedrich-Universität Bamberg übermittelt wurden um ein Pflichtenheft zur Softwareentwicklung (Upgrade zum UnivIS 1.0, welches sich aktuell schon uni-weit im Einsatz befindet) einer Verwaltungsanwendung für Lehrveranstaltungen und Räumen für einen bestimmten Uni-Standort zu erarbeiten.
In diesen Vorgaben, dem Lastenheft, wurden Ergebnisse aus eigens angeführten Ermittlungen zu den individuellen Anforderungen durch Fragebögen, Selbstaufschreibungen sowie Feldbeoachtungen zur aktuellen und gewünschten Einsatzsituation des UnivIS 1.0 und 2.0 zusammengestellt [vgl. \cite{Balzert09}].
Nach der ausführlichen Sichtung aller Unterlagen und einer ersten Vorstudie der Anforderungen konnte vorliegendes Pflichtenheft erstellt werden. In diesem finden Sie jegliche fachliche Definitionen und Anforderungen, die im Zusammenhang mit der angeforderten Software und den gewünschten Funktionen und Leistungen stehen.
\\
\\
Aus diesem Grund werden zunächst die unterschiedlichen Zielbestimmungen, der präzise Produkteinsatz sowie die gesamte Umgebung (Soft-, Hard-, Orgware und Produkt-Schnittstellen) erläutert.
Dem folgen detaillierte Spezifikationen zu den Funktionen, Daten und Leistungen. Um erste Eindrücke zu vermitteln werden Anforderungen an die Benutzeroberfläche sowie erste Prototyp-User-Interfaces vorgestellt. Das Pflichtenheft schließt mit qualitätsbezogenen Zielbestimmungen, globalen Testszeanrien und genauen Angaben zur Entwicklungsumgebung.
Weitere Ergänzungen sowie ein Glossar dienen der Komplettierung und Vermittlung einer besseren Verständlichkeit des vorliegende Dokuments.

\subsection{Definitionen}
\label{sec:Definitionen}

In diesem Abschnitt werden Abkürzungen und Begrifflichkeiten erläutert, die im Pflichtenheft verwendet werden. \\[0.25cm]

\begin{tabular}{p{1.5cm}p{14.5cm}}

	/X0/	& Eine derartige Kennzeichnung wird im Pflichtenheft für die Kennzeichnung von Software-Merkmalen verwendet. Anstelle des "`X"' steht die Abkürzung des jeweiligen Merkmals. "`Z"' kennzeichnet Zielbestimmungen, "`F"' Funktionen, "`D"' Daten, "`L"' Leistungen, "`B"' die Benutzeroberfläche, "`Q"' qualitative Bestimmungen und "`T"' Testszenarien. Anstelle der "0" steht die Nummer des jeweiligen Merkmals.  \\
	/XW0/	& Diese Kennzeichnung entspricht der wie bei "`/X0/"' beschrieben, das zusätzliche "`W"' signalisiert allerdings, dass es sich um ein wünschenswertes Merkmal handelt, dass je nach Entwicklungsaufwand und Dringlichkeit ggf. nicht in der ersten Version der Software implementiert sein wird. \\

\end{tabular}