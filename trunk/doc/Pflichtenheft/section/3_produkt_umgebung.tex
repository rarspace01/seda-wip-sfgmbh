\section{Produktumgebung}
\label{sec:Produktumgebung}

Im folgenden wird die Umgebung beschrieben, in der die spätere Software eingesetzt werden wird.

\subsection{Software}
\label{subsec:software}

Die Software beschreibt die die vorzuhaltende Software-Umgebung um eine problemlose Ausführung des Programms sicherzustellen.\\


Als Betriebssystem kommt Windows 7 (Professional) (x64 - 64bit) zum Einsatz. Dieses hat die neusten Updates vorzuweisen und eine gängige aktuelle Antiviren Lösung installiert zu haben.\\

Als Laufzeitumgebung wird auf die Java Virtual Machine Version 7 von Oracle gesetzt.
Hierzu ist das entsprechende Java Runtime Environment in Version 7 vorzuhalten. Es muss sichergestellt werden, dass die entsprechenden Klassenpfade, sowie Systempfade gesetzt sind, sodass der prozess java als auch javaw im CLI bekannt ist. Darüber hinaus müssen .jar-Dateien mit der JRE verknüpft sein um einen Start der Anwendung per Doppelklick zu gewährleisten. Entsprechende Einstellungen sind notfalls in der Registry vorzunehmen [vgl. Kapitel \ref{subsec:jarreg}].\\

Die Datenbank welche in Verbindung mit der Software eingesetzt wird, stellt PostgreSQL in Version 9.2.1 dar.
Es muss sichergestellt werden, dass die betreffenden PCs welche die neue Univis Software verwenden sollen für die Datenbank freigeschaltet sind (Freigabe der jeweiligen IPs).\\

Zusätzlich muss sichergestellt werden, dass entsprechende Netzwerk-Einstellungen (Firewalls, Router) eine ordnungsgemäße Verbindung zwischen Anwendungs-PC und Datenbank erlauben.\\

Das Programm wird auf die Swing-Komponenten von Java setzten, daher ist es notwendig sicher zus tellen, dass eine entsprechende Shell vorhanden ist.
Als Oberfläche wird auf die Standard Fensteroberfläche von Windows 7 (shell: explorer.exe) gesetzt. Kiosk-Systeme sind als GUI nicht vorgesehen. \\


\subsection{Hardware}
\label{subsec:hardware}

Als Hardware auf der das 

CPU: Intel x86/x65 Architektur - >1Ghz
>=1GB DDR2RAM
HDD: >100MB
Maus \& Tastatur
Netwerkanschluss (für DB Anbindung)

\subsection{Orgware}
Eine entsprechende Netzwerkinfrastruktur muss vorhanden sein um eine Verbindung zur Datenbank herstellen zu können.
Die Datenbank (Postgresql) muss auf einem entsprechenden Server installiert sein und als Dienst laufen. Es muss sicher gestellt werden, dass entsprechende Router und Firewalls konfiguriert sind und eine Verbindung in beide Richtungen ermöglichen.
Es wird davon ausgegangen, dass die Hausverwaltung, sowie die Dozenten entsprechende Informationen zur Räumen und Lehrveranstaltungen korrekt und zeitnah in das System einpflegen um eine sinnvolle Nutzung zu ermöglichen.

\subsection{Produkt-Schnittstellen}
