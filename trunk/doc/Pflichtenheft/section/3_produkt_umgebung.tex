\section{Produktumgebung}
\label{sec:Produktumgebung}

\subsection*{(Denis Hamann)}

Im Folgenden wird die Umgebung beschrieben, in der die finale Software eingesetzt werden soll.
Die Produktumgebung umfasst die Basismaschinen und die Systemumgebung, getrennt nach Software, Hardware und
Produktschnittstellen. Zusätzlich werden im Bereich Orgware die zugehörigen organisatorischen Randbedingungen beschrieben.

\subsection{Software}
\label{subsec:software}

Die Software beschreibt die vorzuhaltende Software-Umgebung um eine problemlose Ausführung des Programms sicherzustellen.\\

\begin{tabular}{p{1.5cm}p{14.5cm}}

	 /PU10/	&  Als Betriebssystem kommt Windows 7 (Professional, x64 - 64bit) zum Einsatz. Dieses hat die neusten Updates vorzuweisen und eine gängige sowie aktuelle Antiviren Lösung installiert zu haben.\\[0.25cm]

\end{tabular}

\begin{tabular}{p{1.5cm}p{14.5cm}}

	 /PU20/	&  Im Bereich der Laufzeitumgebung wird auf die Java Virtual Machine Version 7 von Oracle gesetzt.
Hierzu ist das entsprechende Java Runtime Environment (JRE) in Version 7 vorzuhalten. Es muss sichergestellt werden, dass die entsprechenden Klassenpfade, sowie Systempfade gesetzt sind, sodass der Prozess java als auch javaw im CLI bekannt ist. Darüber hinaus müssen .jar-Dateien mit der JRE verknüpft sein um einen Start der Anwendung per Doppelklick zu gewährleisten. Entsprechende Einstellungen sind notfalls in der Registry vorzunehmen [vgl. Kapitel \ref{subsec:jarreg}].\\[0.25cm]

\end{tabular}

\begin{tabular}{p{1.5cm}p{14.5cm}}

	 /PU30/	&  Die Datenbank welche in Verbindung mit der Software eingesetzt wird, stellt PostgreSQL in Version 9.2.1 dar.
Es muss sichergestellt werden, dass die betreffenden PCs welche die neue UnivIS Software verwenden sollen für die Datenbank freigeschaltet sind (Freigabe der jeweiligen IPs).\\[0.25cm]

\end{tabular}


\begin{tabular}{p{1.5cm}p{14.5cm}}

	 /PU40/	&  Zusätzlich soll sichergestellt werden, dass entsprechende Netzwerk-Einstellungen (Firewalls, Router) eine ordnungsgemäße Verbindung zwischen Anwendungs-PC und Datenbank erlauben.\\[0.25cm]

\end{tabular}


\begin{tabular}{p{1.5cm}p{14.5cm}}

	 /PU50/	&  Das Programm wird auf die Swing-Komponenten von Java setzten, daher ist es notwendig sicher zustellen, dass eine entsprechende Shell vorhanden ist.
Als Oberfläche wird auf die Standard Fensteroberfläche von Windows 7 (shell: explorer.exe) gesetzt. Kiosk-Systeme sind als Benutzeroberfläche nicht vorgesehen.\\[0.25cm]

\end{tabular}


\subsection{Hardware}
\label{subsec:hardware}


\begin{tabular}{p{1.5cm}p{14.5cm}}

	 /PU60/	&  Die Software soll auf IBM-Kompatiblen Computern der Intel x86 Architektur ausgeführt werden.
Um sicher zu gehen, dass die Anwendung ausreichend schnell ausgeführt wird, werden nachfolgende Systemvoraussetzungen empfohlen:\\

	&CPU: 1Ghz\\
	&RAM: 1GB\\
	&HDD: 100MB\\
	&Peripherie: Maus \& Tastatur\\
	&\\
	&Zusätzlich muss sichergestellt werden, dass ein entsprechender Netzwerkanschluss für die Verbindung zum Datenbankserver vorhanden ist.\\[0.25cm]


\end{tabular}

\begin{tabular}{p{1.5cm}p{14.5cm}}

	 /PU70/	&  Entsprechende Netzwerkhardware (Router, Switch, Patchpannel, Firewall, Netzwerkkabel) zur Verbindung der Computer mit dem Datenbankserver werden vorausgesetzt.\\[0.25cm]


\end{tabular}



\subsection{Orgware}
\label{subsec:orgware}

\begin{tabular}{p{1.5cm}p{14.5cm}}

	 /PU80/	&  Für die Entwicklung der Software wird vorausgesetzt, dass jeweils ein involvierter Mitarbeiter der unterschiedlichen Organisationsbereiche der Universität zur Beantwortung von fachspezifischen Fragen zur Verfügung steht. Darüber hinaus ist für Arbeiten vor Ort ein entsprechender Platz zu stellen.\\[0.25cm]


\end{tabular}

\begin{tabular}{p{1.5cm}p{14.5cm}}

	 /PU90/	&  Benutzerhandbücher werden, sofern benötigt, in elektronischer Form (PDF) ausgeliefert. Sie beschreiben die grundlegenden Funktionen der Software.\\[0.25cm]


\end{tabular}

\begin{tabular}{p{1.5cm}p{14.5cm}}

	 /PU100/	&  Der Auftraggeber stellt sicher, dass dem Auftragsnehmer entsprechende Informationen zur Verfügung gestellt werden, um eine ordnungsgemäße Autorisierung und Authentifizierung der Benutzer sicherzustellen.\\[0.25cm]


\end{tabular}

\begin{tabular}{p{1.5cm}p{14.5cm}}

	 /PU110/	&  Der Auftraggeber stellt sicher, dass dem Auftragsnehmer entsprechende Informationen zur Verfügung gestellt werden, um eine ordnungsgemäße Autorisierung und Authentifizierung der Benutzer sicherzustellen.\\[0.25cm]


\end{tabular}

\begin{tabular}{p{1.5cm}p{14.5cm}}

	 /PU120/	&  Die Datenbank (PostgreSQL) muss auf einem entsprechenden Server installiert sein und als Dienst laufen. Es muss sicher gestellt werden, dass entsprechende Router und Firewalls konfiguriert sind und eine Verbindung in beide Richtungen möglich ist, um eine Persistierung der Daten zu ermöglichen.\\[0.25cm]


\end{tabular}

\begin{tabular}{p{1.5cm}p{14.5cm}}

	 /PU130/	&  Es wird davon ausgegangen, dass die Hausverwaltung, sowie die Dozenten entsprechende Informationen zu Räumen und Lehrveranstaltungen korrekt und zeitnah in das System einpflegen um eine sinnvolle Nutzung zu ermöglichen.\\[0.25cm]


\end{tabular}




\subsection{Produkt-Schnittstellen}
\label{subsec:productinterface}

\begin{tabular}{p{1.5cm}p{14.5cm}}

	 /PU140/	&  Das Produkt beinhaltet lediglich eine Schnittstelle zur Datenbank wie in Kapitel \ref{subsec:software} beschrieben. Über diese findet die Persistierung der Daten, sowie die Abfrage komplexer Anfragen statt.
Schnittstellen mit der zukünftig geplanten Weboberfläche, sowie Mobilen Anwendung erfolgt nicht über die Software direkt sondern über die gemeinsame Datenbank. Die dort vorliegende Datenstruktur ist allen weiteren Anwendungen bekannt und ermöglicht so eine reibungslose Interaktion zwischen den Systemen.
Parallele Sitzungen der Software werden ebenfalls über die Datenbank synchron gehalten. D.h. trägt Benutzer1 an Workstation1 eine neue Lehrveranstaltung ein, so ist diese auch in Workstation2 bei Benutzer2 bei entsprechender Einstellung zu sehen. Eine direkte Kommunikation zwischen den einzelnen Instanzen der Software findet nicht statt.\\[0.25cm]


\end{tabular}



\subsection{Zukünftige Entwicklungen}
\label{subsec:zukunftentw}

\begin{tabular}{p{1.5cm}p{14.5cm}}

	 /PU150/	&  Wie bereits im Lastenheft angemerkt ist vorgesehen die Software später über den Standort hinaus zu verwenden.
Neben der Ausweitung der Standorte sollen die Zugriffsmöglichkeiten auf das System um ein Webinterface, sowie eine Mobile Anwendung erweitert werden.
Diese Entwicklungen werden in der erstellten Software berücksichtigt und entsprechend umgesetzt.\\[0.25cm]


\end{tabular}

\begin{tabular}{p{1.5cm}p{14.5cm}}

	 /PU160/	&  Um sicherzustellen, dass die spätere Erweiterung problemlos möglich ist, erhält der Auftraggeber entsprechend eine Dokumentation des Quellcodes, sowie der Datenbank bzw. dessen konkrete Relationen.
Für das Ziel das bestehende UnivIS zu einem späteren Zeitpunkt zu ersetzten sollte dementsprechend ein Change-Management Ansatz gefahren werden, in diesem unter anderem auch die Schulung der späteren Benutzer erfolgen sollte. \\[0.25cm]


\end{tabular}


%Das System soll zunächst ausschließlich am Standort ERBA zum Einsatz
%kommen. Sollte es einer Evaluierung nach einem Zeitraum von vier Semestern
%standhalten, plant die Universität den Einsatz auf weitere Standorte
%auszudehnen sowie den Einsatz im Web und auf mobilen Endgeräten. Fernziel
%ist die Ablösung des bestehenden Systems .