\section{Testszenarien}
\label{sec:Testszenarien}


Damit festgestellt werden kann, ob die Software auch einwandfrei funktioniert, müssen verschiedene Testszenarien durchlaufen werden.
Die Tests werden auf verschiedenen Ebenen ausgeführt. Unterschieden wird hierbei zwischen dem Modultest, Integrationstest, Systemtest und letztendlich dem Abnahmetest.
Im Folgenden werden nun diese Testszenarien genauer beschrieben und dargelegt, wie sie am 
Programm durchgeführt werden.
//
\begin{tabular}{p{1.5cm}p{14.5cm}}
 /T10/	& Im Komponententest, welcher wie auch der nachfolgende Modultest unter White-Box-Tests fällt, beinhaltet das Testen von einzeln abgrenzbaren Teilen (Modulen) des Gesamtprogramms. Das Testen erfolgt bei White-Box-Tests in der Entwicklungsumgebung Eclipse, wo der Quelltext sichtbar ist. Einzelne Klassen, Funktionen und Unterprogramme werden im Modultest dem Tester sequenziell unterworfen und auf Fehlerfreiheit geprüft. Dies umfasst unter anderem eine korrekte Ausgabe von Uhrzeiten. Auch SQL-Abfragen an die Datenbank ob die Passwortabfrage korrekt abgespeichert wurden, soll der Test beinhalten. \\[0.25cm]	 
\end{tabular}
