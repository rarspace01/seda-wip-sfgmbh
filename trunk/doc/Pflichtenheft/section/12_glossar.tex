\section{Glossar}

%\newpage{Abkrzugsverzeichnis}
%\addcontentsline{toc}{chapter}{Abkrzungsverzeichnis}
%%\addcontentsline{toc}{chapter}{\protect{Glossar}}
\begin{acronym}[abkuerzungen2]
\acro{API}{Die API stellt eine dokumentierte Software-Schnittstelle dar, die von anderen Programmen aus genutzt werden kann.}
\acro{CLI}{Command Line Interface - Kommandozeile. Die Kommandozeile ist ein Eingabebereich für die Steuerung einer Software, die typischerweise im Textmodus abläuft.}
\acro{Design Patterns}{Hierbei handelt es sich um Entwurfsmuster, welche bei wiederkehrenden Problemen angewandt werden können.}
\acro{DNS}{Ermöglicht es Klarnamen in numerische IP Adressen (z.B. google-public-dns-a.google.com in 8.8.8.8 umzuwandeln)}
\acro{GUI}{Hierbei handelt es sich um die grafische Benutzeroberfläche.}
\acro{IANA}{Eine Organisation die unter anderem IP-Adressen, Domains und Portnummern regelt.}
\acro{IEEE}{Eine Organisation im Bereich Eltrotechnik und Informatik zur Standardisierung von Techniken, Hardware und Software}
\acro{IP}{Ein Protokoll das für die Vermittlung von Daten dient.}
\acro{ISO}{Eine internationale Vereinigung von Normungsorganisationen.}
\acro{JDBC}{Hierbei handelt es sich um eine Datenbankschnittstelle für Java.}
\acro{LDAP}{Ein Verzeichnisdienst um Abfragen und Modifikationen von Informationen zu erlauben.}
\acro{ODBC}{Hierbei handelt es sich um eine Datenbankschnittstelle von Microsoft}
\acro{RFC}{RFCs sind eine Reihe von technischen und organisatorischen Dokumenten zum Internet, die sie zu einem Standard entwickelt haben.}
\acro{Shell}{Eingabe-Schnittstelle zwischen Computer und Benutzer - meist grafisch}
\acro{SQL}{Eine deskriptive Abfragesprache von Datenbanken.}
\acro{TCP}{Ein verbindungsorientiertes Protokoll, um Daten im Netzwerk zu transportieren.}
\acro{UDP}{Ein verbindungsloses Protokoll, um Daten im Netzwerk zu transportieren.}
\end{acronym}
