\section{Zielbestimmung}
\label{sec:Zielbestimmung}

\subsection*{(Mario Serno)}

\begin{tabular}{p{16.5cm}}
Im Rahmen dieses Kapitels werden die Ziele beschrieben, welche durch den Einsatz des Produktes erreicht werden sollen. Sie gliedern sich in Muss-, Wunsch- und Abgrenzungskriterien. Musskriterien beschreiben Ziele die das Produkt ohne Einschränkung erfüllen muss. Wunschkriterien stellen Leistungen dar, welche für den Einsatzzweck nicht unbedingt erforderlich sind, die das Produkt aber dennoch bieten soll. Abgrenzungskriterien beschreiben Ziele, deren Erreichung bewusst nicht angestrebt wird. Sie dienen lediglich zur Eindämmung der Komplexität [vgl. \cite{balz1996}]  .
\end{tabular}\\[0.25cm]

\begin{tabular}{p{16.5cm}}
Da sich die Anwendungsfunktionen der Nutzer deutlich voneinander unterscheiden, ergeben sich nutzerspezifische Zielsetzungen. Die Nutzer des Systems sind Dozenten, die Universitäts- und Hausverwaltung sowie Studenten.
\end{tabular}

\subsection{Musskriterien}

\begin{tabular}{p{1.5cm}p{14.5cm}}	
	 /ZM10/& Die Dozenten sollen die Lehrveranstaltungen verwalten können. \\[0.25cm]
\end{tabular}

\begin{tabular}{p{1.5cm}p{14.5cm}}	
	 /ZM20/& Die Universitäts- und Hausverwaltung soll die Raumdaten verwalten können. \\[0.25cm]
\end{tabular}

\begin{tabular}{p{1.5cm}p{14.5cm}}	
	 /ZM30/& Die Universitäts- und Hausverwaltung soll die Benutzerdaten aller Nutzer pflegen können. \\[0.25cm]
\end{tabular}

\begin{tabular}{p{1.5cm}p{14.5cm}}	
	 /ZM40/& Die Universitäts- und Hausverwaltung soll den Raumbedarf der Lehrstühle und die Raumplanung koordinieren können.  \\[0.25cm]
\end{tabular}

\begin{tabular}{p{1.5cm}p{14.5cm}}	
	 /ZM50/& Die Universitäts- und Hausverwaltung sowie die Dozenten sollen sich für die spezifischen Aufgaben im System authentifizieren. \\[0.25cm]
\end{tabular}

\begin{tabular}{p{1.5cm}p{14.5cm}}	
	 /ZM60/& Alle Nutzer des Systems sollen individuelle Wochenpläne (für Räume und Lehrveranstaltungen) für den Universitätsstandort Erba erstellen können (Ausgabe i. F. einer PDF). \\[0.25cm]
\end{tabular}

\begin{tabular}{p{1.5cm}p{14.5cm}}	
	 /ZM70/& Es soll möglich sein, das System um die restlichen Universitätsstandorte, Lehrveranstaltungen und Angehörige der Universität nach einem Evaluationszeitraum erweitern zu können. \\[0.25cm]
\end{tabular}

\begin{tabular}{p{1.5cm}p{14.5cm}}	
	 /ZM80/& Der Nutzer soll einen Überblick über aktuell laufende und demnächst startende Lehrveranstaltungen erhalten können (Live-Ticker). \\[0.25cm]
\end{tabular}

\begin{tabular}{p{1.5cm}p{14.5cm}}	
	 /ZM90/& Alle Nutzer sollen grundlegende Informationen zu Räumen und Lehrveranstaltungen erhalten können. \\[0.25cm]
\end{tabular}

\begin{tabular}{p{1.5cm}p{14.5cm}}	
	 /ZM100/& Das System soll intuitiv bedienbar sein. \\[0.25cm]
\end{tabular}

\subsection{Wunschkriterien }


\begin{tabular}{p{1.5cm}p{14.5cm}}	
	 /ZW10/& Alle Nutzer des Systems sollten organisatorische Änderungen kurzfristig erkennen können. \\[0.25cm]
\end{tabular}

\begin{tabular}{p{1.5cm}p{14.5cm}}	
	 /ZW20/& Für Dozenten sollte ein personalisierter Live-Ticker zur Verfügung stehen. \\[0.25cm]
\end{tabular}

\begin{tabular}{p{1.5cm}p{14.5cm}}	
	 /ZW30/& Für Studenten sollte ein personalisierter Login-Bereich bereitgestellt werden um die Sammlung der Lehrveranstaltungen persistent speichern zu können sowie einen personalisierten Live-Ticker zu integrieren. \\[0.25cm]
\end{tabular}

\subsection{Abgrenzungskriterien}

\begin{tabular}{p{1.5cm}p{14.5cm}}	
	 /ZA10/& Studenten soll kein personalisierter Login zur Verfügung stehen.  \\[0.25cm]
\end{tabular}

\begin{tabular}{p{1.5cm}p{14.5cm}}	
	 /ZA20/& Von Studenten erstellte Sammlungen für Wochenpläne sollen nicht über die Dauer der Systemnutzung hinaus gespeichert werden. \\[0.25cm]
\end{tabular}