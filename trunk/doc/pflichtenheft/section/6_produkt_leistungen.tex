%-------------------------------
\section{Leistungen}
%-------------------------------


\begin{tabular}{p{1.5cm}p{14.5cm}}


	 /L10/	& Alle Daten, die im vorherigen Abschnitt aufgelistet sind, müssen, sofern sie realisiert werden, persistent mittels einer SQL-Datenbank (erreichbar und verwaltet durch einen PostgreSQL-Server) gespeichert werden. \\[0.25cm]
	 
\end{tabular}


\begin{tabular}{p{1.5cm}p{14.5cm}}
						
	 /L20/	& Bei allen Prozess mit Dateneingabe-Funktionen erhalten Nutzer aussagekräftige Fehlermeldungen, sollte ihr Prozess nicht ordnungsgemäß durchgeführt werden können. \\[0.25cm]
	 
\end{tabular}


\begin{tabular}{p{1.5cm}p{14.5cm}}
						
	 /L30/	& Die Realisierung erfolgt als Java-Anwendung, so dass ein möglicher Betrieb auf allen gängigen PCs der Universität sichergestellt ist. \\[0.25cm]

\end{tabular}


\begin{tabular}{p{1.5cm}p{14.5cm}}
						
	 /L40/	& Die Programmarchitektur richtet sich nach dem MVC-Modell, wobei eine saubere untergliederung zwischen den Schichten gewährleistet wird, sowie ein hohes Maß an Modularität erreicht wird. Dieses erlaubt es auch verhältnismäßig einfach Erweiterungen zu einem späteren Zeitpunkt zu erstellen. \\
	 &	Die erste Version der Anwendung soll aus folgenden Modulen bestehen: \\
	 &	\textbf{CoreGUI:} Hier wird die Start-Routine sowie der Startbildschirm der Anwendung realisiert. Die Funktion /F01/ und /F60/ werden von diesem Modul umgesetzt.\\
	 &	\textbf{Verwaltung:} Hier wird die Anzeige für Verwaltungsmitglieder umgesetzt, die die Funktion /F10/, wünschenswerterweise /FW21/, /F40/, /F50/, /F51/, /F80/ und /F90/ realisiert.\\
	 &	\textbf{Dozenten:} Hier wird die Anzeige für Dozenten umgesetzt, die die Funktion /F20/, wünschenswerterweise /FW21/, /F30/, /F140/, /F150/ und /F70/ realisiert.\\
	 &	\textbf{Stundenplan:} Hier wird die Anzeige für den Stundenplan von Studenten umgesetzt, die die Funktion /F130/ realisiert.\\
	 &	\textbf{Raumplan:} Hier wird die Anzeige für Mitglieder der Verwaltung (und ggf. andere) von Räumen und deren Belegung umgesetzt, die die Funktion /F120/ realisiert.\\
	 &	\textbf{Studentenprofil (wünschenswert):} Hier wird die Anzeige für Studenten umgesetzt, die die Funktion /FW61/ realisiert.\\[0.25cm]

\end{tabular}


\begin{tabular}{p{1.5cm}p{14.5cm}}
					
	 /L50/	& Es wird eine Nutzerauthentifizierung realisiert, die es gewährleistet, dass nur berechtigte Nutzer Änderungen durchführen. Um Missbrauch ausschließen zu können, sollte allerdings eine Middleware (die großteils die Model- und Controller-Schicht abdeckt) zwischen Datenbank und Client (der haupsächlich die View-Schicht umsetzt) realisiert werden. In diesem Fall sollte die Kommunikation zwischen Client und Middleware durch SSL (mithilfe von javax.net.ssl.* und javax.rmi.ssl.*) abgesichert werden. \\[0.25cm]
	
\end{tabular}


\begin{tabular}{p{1.5cm}p{14.5cm}}
					
	 /L60/	& Zeitliche Spezifikationen (Abgleich mit Christian) \\[0.25cm]
	
\end{tabular}



%################################
\end{document}
%################################