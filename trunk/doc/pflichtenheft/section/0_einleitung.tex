\section{Einleitung}
\label{sec:Einleitung}

Die Vorliegende Arbeit enthält die Gesamtheit an notwendigen Spezifikationen die uns von der Otto-Friedrich-Universität übermittelt wurden. 
In diesen Vorgaben, dem Lastenheft, wurden Ergebnisse aus eigens angeführten Ermittlungen zu den individuellen Anforderungen durch Fragebögen, Selbstaufschreibungen sowie Feldbeoachtungen [Balz09, S.303] zur aktuellen und gewünschten Einsatzsituation der neuen Software UniVis 2.0 zusammengestellt.
Nach der ausführlichen Sichtung aller Unterlagen und einer ersten Vorstudie der Anforderungen konnte dieses Pflichtenheft erstellt werden. In diesem finden Sie jegliche fachliche Definitionen und Anforderungen, die im Zusammenhang mit der gewünschten Software UniVis 2.0 und den gewünschten Funktionen und Leistungen stehen.
\\
\\
Aus diesem Grund werden zunächst die unterschiedlichen Zielbestimmungen, der präzise Produkteinsatz sowie die gesamte Umgebung (Soft-, Hard-, Orgware und Produkt-Schnittstellen) erläutert.
Dem folgen detaillierte Spezifikationen zu den Funktionen, Daten und Leistungen. Um erste Eindrücke zu sammeln und somit zu vermitteln werden Anforderungen an die Benutzeroberfläche sowie erste Prototyp-User-Interfaces vorgestellt. Das Pflichtenheft schließt mit qualitätsbezogenen Zielbestimmungen, globalen Testszeanrien und genauen Angaben zur Entwicklungsumgebung.
Weiter Ergänzungen sowie ein Glossar dienen der Komplettierung und Vermittlung einer besseren Verständlichkeit des vorliegende Dokuments.

