\section{Produkteinsatz}
\label{sec:Produkteinsatz}
Das Produkt dient der Verwaltung von Räumen und Lehrveranstaltungen am Standort ERBA der Otto-Friedrich-Universität Bamberg. Insbesondere soll der Raumbedarf der Lehrstühle und die Raumplanung durch die Hausverwaltung effektiv koordiniert werden. Des Weiteren soll den Studenten eine Möglichkeit geboten werden, sich einen Stundenplan zu erstellen.

\subsection{Anwendungsbereiche}
\begin{tabular}{p{1.5cm}p{14.5cm}}	
	 /P10/& Das Produkt wird an der Otto-Friedrich-Universität Bamberg intern eingesetzt. \\[0.25cm]
\end{tabular}


\subsection{Zielgruppen}
\begin{tabular}{p{1.5cm}p{14.5cm}}	
	 /P20/& Zielgruppe des Produktes ist der Lehrstuhl für Systementwicklung und Datenbankanwendung der Otto-Friedrich-Universität Bamberg. \\[0.25cm]
\end{tabular}


\subsection{Betriebsbedingungen}
\begin{tabular}{p{1.5cm}p{14.5cm}}	
	 /P30/& Das Produkt wird auf Clients der Hausverwaltung, Lehrstühle und PC-Pools ausgeführt. \\[0.25cm]
\end{tabular}

\begin{tabular}{p{1.5cm}p{14.5cm}}	
	 /P30/& Das System ist so zu konzipieren, dass die einzelnen Anwendungsinstanzen über eine gemeinsame Datenbasis (PostgreSQL-Server) kommunizieren können. \\[0.25cm]
\end{tabular}

\begin{tabular}{p{1.5cm}p{14.5cm}}	
	 /P30/& Die tägliche Betriebszeit des Produktes erstreckt werktags jeweils von 08:00 Uhr bis 20:00 Uhr. Eine Nutzung am Wochenende ist nicht vorgesehen. \\[0.25cm]
\end{tabular}

\begin{tabular}{p{1.5cm}p{14.5cm}}	
	 /P30/& Zu Beginn jedes Semesters wird die Lauffähigkeit des Produktes durch einen Administrator über einen Zeitraum von drei Wochen beobachtet. Nach Ablauf dieses Zeitraums ist ein unbeaufsichtigter Betrieb vorgesehen. \\[0.25cm]
\end{tabular}
