\section{Produkteinsatz}
\label{sec:Produkteinsatz}

\subsection*{(Mario Serno)}

Das Produkt dient der Verwaltung von Räumen, Lehrveranstaltungen und Systemnutzer am Standort Erba der Otto-Friedrich-Universität Bamberg. Insbesondere soll der Raumbedarf der Lehrstühle und die Raumplanung durch die Hausverwaltung effektiv koordiniert werden. Des Weiteren soll den Studenten eine Möglichkeit geboten werden, sich einen Stundenplan zu erstellen.

\subsection{Anwendungsbereiche}
\begin{tabular}{p{1.5cm}p{14.5cm}}	
	 /P10/& Das Produkt wird intern an der Otto-Friedrich-Universität Bamberg eingesetzt. \\[0.25cm]
\end{tabular}


\subsection{Zielgruppen}
\begin{tabular}{p{1.5cm}p{14.5cm}}	
	 /P20/& Zielgruppen des Produktes sind Lehrstühle, deren Mitarbeiter, die Hausverwaltungsmitarbeiter und Studenten der der Otto-Friedrich-Universität Bamberg am Standort Erba. \\[0.25cm]
\end{tabular}


\subsection{Betriebsbedingungen}
\begin{tabular}{p{1.5cm}p{14.5cm}}	
	 /P30/& Das Produkt wird auf den Clients der Hausverwaltung, Lehrstühle und PC-Pools ausgeführt. \\[0.25cm]
\end{tabular}

\begin{tabular}{p{1.5cm}p{14.5cm}}	
	 /P40/& Das System ist so zu konzipieren, dass die einzelnen Anwendungsinstanzen über eine gemeinsame Datenbasis (PostgreSQL-Server) kommunizieren können. \\[0.25cm]
\end{tabular}

\begin{tabular}{p{1.5cm}p{14.5cm}}	
	 /P50/& Die tägliche Betriebszeit des Produktes erstreckt sich werktags jeweils von 08:00 bis 20:00 Uhr. Eine Nutzung am Wochenende ist nicht vorgesehen. \\[0.25cm]
\end{tabular}

\begin{tabular}{p{1.5cm}p{14.5cm}}	
	 /P60/& Zu Beginn jedes Semesters wird die Lauffähigkeit des Produktes durch einen Administrator über einen Zeitraum von drei Wochen beobachtet. Nach Ablauf dieses Zeitraums ist ein unbeaufsichtigter Betrieb vorgesehen. \\[0.25cm]
\end{tabular}
