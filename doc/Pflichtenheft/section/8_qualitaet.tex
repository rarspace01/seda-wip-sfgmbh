\section{Qualitätszielbestimmung}
\label{sec:Qualitätszielbestimmung}

\subsection*{(Christian Hindelang)}

Das zu entwerfende Programm (UnivIS 2.0) hat folgende Qualitätsziele, die nachfolgend erläutert werden [vgl. \cite{UniRos12a}]. 

\begin{tabular}{p{1.5cm}p{14.5cm}}
 /Q10/	& Das Programm soll eine korrekte Funktionalität aufweisen, d. h. es soll richtige Ergebnisse bei den Ein- und Ausgaben liefern. \\[0.25cm]	 
\end{tabular}

\begin{tabular}{p{1.5cm}p{14.5cm}}
 /Q20/	& Die Software soll zuverlässig arbeiten. Es sollten möglichst keine Fehler passieren. \\[0.25cm]	 
 \end{tabular}

\begin{tabular}{p{1.5cm}p{14.5cm}}
 /Q30/	& Es soll gewährleistet sein, dass die Daten sicher verwahrt sind und vor unberechtigtem Zugriff geschützt sind. \\[0.25cm]	 
\end{tabular}

\begin{tabular}{p{1.5cm}p{14.5cm}}
 /Q40/	& Im falle eines die Datenrückgewinnung soll gewährleistet sein. \\[0.25cm]	 
\end{tabular}

\begin{tabular}{p{1.5cm}p{14.5cm}}
 /Q50/	& Eine leichte und intuitive Bedienung von UnivIS 2.0 ist von Nöten, da eine Vielzahl von potenziellen Nutzern auf das System zugreifen können sollen. Die Heterogenität der Studenten und des Personals macht es erforderlich, dass sich alle Beteiligten schnell an die Bedienung des Programms gewöhnen und keine langen Einarbeitungsphasen oder sogar Kurse notwendig werden. \\[0.25cm]	 
\end{tabular}

\begin{tabular}{p{1.5cm}p{14.5cm}}
 /QW60/	& Damit die Nutzer des Systems effizient arbeiten können, soll im Mittel die Ergebnisausgabe eines Stundenplans drei Sekunden in Anspruch nehmen. \\[0.25cm]	 
\end{tabular}

\begin{tabular}{p{1.5cm}p{14.5cm}}
 /Q70/	& Die im Lastenheft gemachte Angabe, dass das System eventuell universitätsweit und auf mobilen Plattformen zum Einsatz kommen kann, macht es erforderlich, dass die Änderbarkeit und Anpassung des Programms schnell erfolgen kann. Maximal drei Monate sollten hierfür aufgewendet werden müssen. \\[0.25cm]	 
\end{tabular}
 
\begin{tabular}{p{1.5cm}p{14.5cm}}
 /Q80/	& Schlussendlich muss das Programm an andere Windowsversionen anpassbar sein. Dies ist durch eine .java-Datei gewährleistet welche plattformunabhängig läuft. \\[0.25cm]	 
\end{tabular} 
 