\section{Qualitätszielbestimmung}
\label{sec:Qualitätszielbestimmung}

Das zu entwerfende Programm (UnivIS 2.0) hat folgende Qualitätsziele, die nachfolgend erläutert werden. 

\begin{tabular}{p{1.5cm}p{14.5cm}}
 /Q10/	& Das Programm muss eine korrekte Funktionalität aufweisen, d. h. es muss richtige Ergebnisse bei Eingaben und Ausgaben liefern. \\[0.25cm]	 
\end{tabular}

\begin{tabular}{p{1.5cm}p{14.5cm}}
 /Q20/	& Es muss gewährleistet sein, dass die Daten sicher und vom Zugriff unberechtigter geschützt sind. \\[0.25cm]	 
\end{tabular}

\begin{tabular}{p{1.5cm}p{14.5cm}}
 /Q30/	& Die Software muss zuverlässig arbeiten. Es dürfen möglichst keine Fehler passieren. \\[0.25cm]	 
\end{tabular}

\begin{tabular}{p{1.5cm}p{14.5cm}}
 /Q40/	& Die Datenrückgewinnung muss gewährleistet sein, wenn der Server abstürzt. \\[0.25cm]	 
\end{tabular}

\begin{tabular}{p{1.5cm}p{14.5cm}}
 /Q50/	& Eine leichte und intuitive Bedienung von UniVis 2.0 ist von Nöten, da eine Vielzahl von potenziellen Nutzern auf das System zugreifen soll. Die Heterogenität der Studenten und des Personals macht es erforderlich, dass sich alle Beteiligten schnell an die Bedienung des Programms gewöhnen und keine langen Einarbeitungsphasen oder sogar Kurse notwendig werden. \\[0.25cm]	 
\end{tabular}

\begin{tabular}{p{1.5cm}p{14.5cm}}
 /Q60/	& Damit die Nutzer des Systems effizient arbeiten können, darf im Mittel die Ergebnisausgabe eines Stundenplans zwei Sekunden in Anspruch nehmen. \\[0.25cm]	 
\end{tabular}

\begin{tabular}{p{1.5cm}p{14.5cm}}
 /Q60/	& Die im Lastenheft gemachte Angabe, dass das System eventuell universitätsweit und auf mobilen Plattformen zum Einsatz kommen kann, macht es erforderlich, dass die Änderbarkeit und Anpassung des Programms schnell erfolgen kann. Maximal drei Monate sollten hierfür aufgewendet werden müssen. \\[0.25cm]	 
\end{tabular}
 
\begin{tabular}{p{1.5cm}p{14.5cm}}
 /Q60/	& Schlussendlich muss das Programm an andere Windowsversionen anpassbar sein. Dies ist gewährleistet, da eine JAVA-Datei verwendet wird, welche plattformunabhängig läuft. \\[0.25cm]	 
\end{tabular} 
 