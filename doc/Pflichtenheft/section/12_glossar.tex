\section{Glossar}

%\newpage{Abkrzugsverzeichnis}
%\addcontentsline{toc}{chapter}{Abkrzungsverzeichnis}
%%\addcontentsline{toc}{chapter}{\protect{Glossar}}
\begin{acronym}[abkuerzungen2]
\acro{ADO}{Hierbei handelt es sich um eine Datenbankschnittstelle von Microsoft}
\acro{API}{Die API stellt eine dokumentierte Software-Schnittstelle dar, die von anderen Programmen aus genutzt werden kann.}
\acro{Apple-Talk}{Ein Protokoll von Apple, das für die Vermittlung von Daten dient.}
\acro{ASP}{Active Server Pages ist ein Standard der Firma Microsoft, um dynamische Webseiten zu erzeugen.}
\acro{CDP}{Ein Cisco Protokol zur Kommunikation der Cisco Geräte untereinander}
\acro{CSMA}{CSMA dient zur Erkennung und Behandlung von Kollisionen auf einem Medium.}
\acro{Design Patterns}{Hierbei handelt es sich um Entwurfsmuster, welche bei wiederkehrenden Problemen angewandt werden können.}
\acro{DNS}{Ermöglicht es Klarnamen in numerische IP Adressen (z.B. google-public-dns-a.google.com in 8.8.8.8 umzuwandeln)}
\acro{Duplex-Mismatch}{Fehler bei der Duplex-Aushandlung}
\acro{ERM}{Dient zur Datenmodelierung von Datenbanken}
\acro{GPL}{GPL ist eine Open Source-Softwarelizenz.}
\acro{GUI}{Hierbei handelt es sich um die grafische Benutzeroberfläche.}
\acro{IANA}{Eine Organisation die unter anderem IP-Adressen, Domains und Portnummern regelt.}
\acro{IEEE}{Eine Organisation im Bereich Eltrotechnik und Informatik zur Standardisierung von Techniken, Hardware und Software}
\acro{IETF}{Eine Organisation die sich mit der technischen Weiterentwicklung des Internets befasst.}
\acro{IIS}{Beim ISS handelt es sich um einen Webserver von Microsoft.}
\acro{IP}{Ein Protokoll das für die Vermittlung von Daten dient.}
\acro{IPX}{Ein Protokoll das für die Vermittlung von Daten dient.}
\acro{ISO}{Eine internationale Vereinigung von Normungsorganisationen.}
\acro{IOS}{Ein Betriebsystem das auf Cisco Router und Switches läuft.}
\acro{JDBC}{Hierbei handelt es sich um eine Datenbankschnittstelle für Java.}
\acro{JSP}{JavaServer Pages ist ein Standard der Firma Oracle, um dynamische Webseiten zu erzeugen.}
\acro{LDAP}{Ein Verzeichnisdienst um Abfragen und Modifikationen von Informationen zu erlauben.}
\acro{MAC}{MAC kontrolliert den Zugriff auf ein Netzwerkmedium. }
\acro{MIB}{Bei einer MIB handelt es sich um Beschreibungen von Daten, die per Netzwerk ausgetausch oder modifiziert werden.}
\acro{MRTG}{Ein Program zur grafischen Auswertung oder Darstellung von Messwerten, vor allem im Netzwerkbereich.}
\acro{Nagios}{Ein Programm zum Abfragen von Diensten und Hosts.}
\acro{ODBC}{Hierbei handelt es sich um eine Datenbankschnittstelle von Microsoft}
\acro{OID}{Ein eindeutiger Bezeichner eines Objektes.}
\acro{PHP}{Eine Programmiersprache um dynamische Webseiten zu erzeugen.}
\acro{RFC}{RFCs sind eine Reihe von technischen und organisatorischen Dokumenten zum Internet, die sie zu einem Standard entwickelt haben.}
\acro{SNMP}{Ein Netzwekrprotokoll, um Netzwerkelemente zu überwachen und steuern.}
\acro{SQL}{Eine deskriptive Abfragesprache von Datenbanken.}
\acro{SSH}{Ein Protokoll um eine verschlüsselte Netzwerkverbindung aufzubauen, um z.B. ein System fernzusteuern.}
\acro{STP}{Ein Protokoll zwischen Switches um Broadcaststürme zu vermeiden.}
\acro{TCP}{Ein verbindungsorientiertes Protokoll, um Daten im Netzwerk zu transportieren.}
\acro{UDP}{Ein verbindungsloses Protokoll, um Daten im Netzwerk zu transportieren.}
\acro{UML}{Eine grafische Modellierungsprache zur Erstellung und Dokumentation von Software.}
\acro{VLAN}{Ermöglicht eine logische Unterteilung eines bestehenden Netzwerkes.}
\acro{VTP}{Ein Protokoll zum definieren von VLAN-Trunking Ports}
\acro{VOIP}{Voice over IP bezeichnet eine Technologie, die es ermöglicht per Netzwerkprotokoll Telefonverbindungen herzustellen}
\acro{WAN}{Bei einem Wide Area Network handelt es sich um ein Netzwerk, bei dem die Endgeräte weit voneinander entfernt sind.}
\end{acronym}
