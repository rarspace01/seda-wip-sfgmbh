\section{Benutzeroberfläche}
\label{sec:Benutzeroberfläche}

Dieser Abschnitt beschäftigt sich mit der grundlegenden Gestaltung der Benutzeroberfläche für die unterschiedlichen Nutzer. Die Anforderungen sind untergliedert in unabdingbare Anforderungen (/BXX/) sowie wünschenswerte (/BFXX/).
\subsection{Bildschirmlayout}
/B10/ Übersichtliche Gestaltung der Funktionen
/B11/ Standardmäßig startet das Programm mit einer Suchmaske. Weitere Funktionen sind via Tabs und Anmeldung erreichbar.Siehe Abbildung:

/B20/ Anpassbarkeit der Fenstergröße
/B30/ Der Nutzer interagiert mit der Benutzeroberfläche über Tastatur- und Mauseingaben.
\subsection{Drucklayout}
/B40/ Nicht Angemeldete Nutzer können über die Auswahl verschiedener Lehrveranstaltungen einen Stundenplan im PDF-Format erzeugen lassen. Siehe Abbildung:

/B41/ Die Hausverwaltungsnutzer können folgende Raumpläne im PFD-Format erzeugen lassen:

/B42/ Mitarbeiter der Lehrstühle können personen- oder lehrstuhlbezogene Wochenpläne in ein PDF-Format exportieren.Siehe Abbildung:
 
\subsection{Tastaturbelegung}
eigenlich keine...
\subsection{Dialogstruktur}
\subsection{Bildschirmlayout}

Entwurfsskizzen aller Bildschirme, Beschreibung der Bedienung
Zuordnung von funktionen, wenn spezielle Tasten belegt werden
Beschreibung in welcher Weise dei kommunikation zwischen benutzer und DV-System erfolgen soll
 /B30/ DIN 66234, Teil 8 ist zu beachten.
 /B40/ Fensterlayout, Dialogstruktur und Mausbedienung 
entsprechen dem Windows-Gestaltungs-Regelwerk (style guide) p g g(yg)
 /B50/ Sämtliche Daten sind passwortgeschützt und dürfen nur 
von autorisierten Mitarbeitern des Lehrstuhls bearbeitet werden.