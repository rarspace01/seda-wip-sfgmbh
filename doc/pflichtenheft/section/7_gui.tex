\section{Benutzeroberfläche}
\label{sec:Benutzeroberfläche}


Dieser Abschnitt beschäftigt sich mit der grundlegenden Gestaltung der Benutzeroberfläche für die unterschiedlichen Nutzer. Die Anforderungen sind untergliedert in unabdingbare Anforderungen (/BXX/) sowie wünschenswerte (/BFXX/).
//

\begin{tabular}{p{1.5cm}p{14.5cm}}
 /B10/	& Fensterlayout, Dialogstruktur und Mausbedienung entsprechen dem Windows-Gestaltungs-Regelwerk. \\[0.25cm]	 
\end{tabular}

\begin{tabular}{p{1.5cm}p{14.5cm}}
 /B11/	& Sämtliche Daten sind passwortgeschützt und dürfen nur 
von autorisierten Mitarbeitern des Lehrstuhls bearbeitet werden. \\[0.25cm]	 
\end{tabular}


\subsection{Bildschirmlayout}

\begin{tabular}{p{1.5cm}p{14.5cm}}
 /B20/	& Übersichtliche Gestaltung der Funktionen. \\[0.25cm]	 
\end{tabular}

\begin{tabular}{p{1.5cm}p{14.5cm}}
 /B30/	& Standardmäßig startet das Programm mit einer Suchmaske. Weitere Funktionen sind via Tabs und Anmeldung erreichbar. \\[0.25cm]	 
\end{tabular}

\begin{tabular}{p{1.5cm}p{14.5cm}}
 /BF40/	& Individuelle Anpassbarkeit der Fenstergröße soll möglich sein. \\[0.25cm]	 
\end{tabular}

\subsection{Drucklayout}

\begin{tabular}{p{1.5cm}p{14.5cm}}
 /B50/	& Nicht Angemeldete Nutzer können über die Auswahl verschiedener Lehrveranstaltungen einen Stundenplan im PDF-Format erzeugen lassen. \\[0.25cm]	 
\end{tabular}

\begin{tabular}{p{1.5cm}p{14.5cm}}
 /B60/	& Die Hausverwaltungsnutzer können folgende Raumpläne im PFD-Format erzeugen lassen. \\[0.25cm]	 
\end{tabular}

\begin{tabular}{p{1.5cm}p{14.5cm}}
 /B70/	& Mitarbeiter der Lehrstühle können personen- oder lehrstuhlbezogene Wochenpläne in ein PDF-Format exportieren in einem übersichtlchen A4 Format. \\[0.25cm]	 
\end{tabular}
 
\subsection{Tastaturbelegung}

\begin{tabular}{p{1.5cm}p{14.5cm}}
 /BF80/	& Mögliche individuelle, nicht dem Windows Standard entsprechende, Tastaturbelegungen sind ein Wunschkriterium. \\[0.25cm]	 
\end{tabular}

Zuordnung von funktionen, wenn spezielle Tasten belegt werden

\subsection{Dialogstruktur}

\begin{tabular}{p{1.5cm}p{14.5cm}}
 /B90/	& Zu Beachten ist: ISO 9241-10: 1996 bzgl. der Ergonomischen Anforderungen für Bürotätigkeiten mit Bildschirmgeräten, Teil 10:Grundsätze der Dialoggestaltung. \\[0.25cm]	 
\end{tabular}

\begin{tabular}{p{1.5cm}p{14.5cm}}
 /B91/	& Folgende Rollen sind zu unterscheiden: \\[0.25cm]	 
\end{tabular}\\

\begin{table}
\begin{tabular}{l|l}
Rolle&Rechte\\
\hline
\hline
Student & (F01),F60 , FW61 \\
\hline
Dozent & F01, F20, FW21, F30  \\
\hline
Verwaltungsangestellter & F01, F10, FW21, F40, F50, F51,  \\
\hline
\end{tabular}
\end{table}

\subsection{GUI-Prototyp}

Entwurfsskizzen aller Bildschirme, Beschreibung der Bedienung
Beschreibung in welcher Weise die kommunikation zwischen benutzer und DV-System erfolgen soll

